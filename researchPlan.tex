\documentclass[10pt, oneside]{article}   	
\usepackage[utf8]{inputenc}
\usepackage{amssymb}


\title{Research stay plan}
\author{Muzaffar Artikov\\
Software Engineering Department\\
Urgench branch of Tashkent University of Information Technologies}
\date{\today}							

\usepackage{xcolor}
\usepackage{mdframed}
\usepackage{booktabs}
\usepackage{array,graphicx}
\newmdtheoremenv[
	backgroundcolor=gray!15,
	roundcorner=5pt,
	innertopmargin=\baselineskip,
    skipabove=\baselineskip,
    aftersingleframe=\bigskip,
    afterlastframe=\bigskip]
{definition}{\textbf{Definition}}[section]

\begin{document}
\maketitle
My name is Muzaffar Artikov. I am currently working at the Urgench branch of Tashkent University of Information Technologies (TUIT). Simultaneously, I am doing my PhD research in the field of Software Engineering. 

\section{Academic background}

After finishing my bachelors study in 2013 and masters study in 2015, I have continued working at TUIT Urgench branch as a junior lecturer. I give lectures on the subjects of “Web applications’ development”, “Software architectures, designing and development”, “Embedded Software Systems”, which are strongly related to the area of my interest. During these teaching activities, I worked on several projects as a software architect, database designer and developer in the framework of the “e-Cardio” project developed for the Cardiology Center in Khorezm (Uzbekistan) in 2015-2017 years. Besides, I took a part in development of the “InfoSecurity” project as a developer in 2016. By participating in these projects, I have expanded my knowledge in the Software Engineering field. Teaching and developing activities strengthened both my theoretical and practical knowledge in Software Engineering. 
\section{Motivation}

Today’s technological advancements leading to an increasingly interconnected world. We are using smarter devices and services in our everyday life. Internet of things (IoT) has become the new technological trend today. Therefore, I have chosen model-driven IoT as my PhD research area. I am currently working in the PhD topic entitled "Model-driven Internet of Things".
German universities, and particularly, University of Oldenburg is one of the leading institutions in this field. I want to do my PhD work in the Department of Software Engineering at the University of Oldenburg. I am interested in the scientific works of this department, since they are closer to my interests. Additionally, both the University of Oldenburg and TUIT have signed a "Memorandum of Understanding" in order to setup and strengthen partnership and cooperation in Software 
Engineering and it’s application areas. 

I plan to expand my knowledge in the field of software engineering while visiting University of Oldenburg. In particular, to strengthen the theoretical knowledge of the foundations of modeling, model-driven architectures, version-control systems, service-oriented architectures, microservice architectures and modeling for IoT topics. The main purpose of research stay is to complete my research proposal in the field of Model-driven IoT. At the same time, I intend to participate in current projects of Software engineering group of University of Oldenburg. Participation in these projects will help me to enrich my practical skills in software engineering.

In order to develop my scientific work and establish strong ties with the department, I would like to visit the University of Oldenburg and closer get acquainted with my supervisor and Software Engineering group. This research stay will not be paid by TUIT and i will have to finance my stay in Oldenburg. Thereby, it would be very helpful for me if i am awarded for the joint contact scholarship for foreign doctoral students by DAAD. Since I have no other source of financial support except my own budget, it would be very helpful for my research stay in Oldenburg and develop closer cooperations with Prof. Dr. Andreas Winter.
I am also very keen on developing and expanding further cooperation between the Tashkent University of Information Technologies and the Carl von Ossietzky University of Oldenburg. 

\end{document}