\documentclass[10pt, oneside]{article}   	
\usepackage[utf8]{inputenc}
\usepackage{amssymb}


\title{Research Plan}
\author{Muzaffar Artikov\\
Software Engineering Department\\
Urgench branch of Tashkent University of Information Technologies}
\date{\today}							

\usepackage{xcolor}
\usepackage{mdframed}
\usepackage{booktabs}
\usepackage{array,graphicx}
\newmdtheoremenv[
	backgroundcolor=gray!15,
	roundcorner=5pt,
	innertopmargin=\baselineskip,
    skipabove=\baselineskip,
    aftersingleframe=\bigskip,
    afterlastframe=\bigskip]
{definition}{\textbf{Definition}}[section]

\begin{document}
\maketitle

\section{Academic Background}
I am currently working at the Urgench branch of Tashkent University of Information Technologies (TUIT). Simultaneously, I am doing my PhD research in the field of Software Engineering. 

After finishing my bachelors study in 2013 and masters study in 2015, I have continued working at TUIT Urgench branch as a junior lecturer. I give lectures on the subjects of "Web Application Development", "Software Architectures, Designing and Development", "Embedded Software Systems", which are strongly related to the research area of my interest. During these teaching activities, I worked on several projects as a software architect, database designer and developer in the framework of the "e-Cardio" project developed for the Cardiology Center in Khorezm (Uzbekistan) in 2015-2017 years. Besides, I took a part in development of the "InfoSecurity" project as a developer in 2016. By participating in these projects, I have expanded my knowledge in the Software Engineering field. Teaching and developing activities strengthened both my theoretical and practical knowledge in Software Engineering.

\section{Existing Collaborations Between Partner Institutes}
Initial contacts between the partner institutes have been started by scholarships, starting in 2011 in the framework of the TARGET project of Erasmus Mundus. In 2014, a memorandum of understanding between UOL and TUIT was signed, to establish a formal framework for future cooperation (see Appendix). Motivated by recent efforts to establish a software engineering institute at TUIT, Dr. Bakhtiyorjon Akbaraliev (department head) and Oybek Allamov, visited UOL’s software engineering group for the 2014/2015 winter term. This stay was used to exchange and transfer knowledge in software engineering research and teaching.

To extend existing cooperations, the parter institutes have submitted a collaboration project proposal entitled \textbf{"Smart Modeling"} to the BMBF call (BMBF Bekanntmachung: "Richtlinien zur Förderung der Wissenschaftlich-Technische Zusammenarbeit mit der Republik Usbekistan: Vorhabenbeschreibung") in 2016. The project proposal is initially received a good notification from the grant provider. The existing memorandum of understanding was also extended in 2017, accordingly. Furthermore, the both universities together with other partners have submitted another project proposal in the field of \textbf{"Smart City"} to another BMBF call (Bekanntmachung: Richtlinie zur Förderung von Vernetzungs- und Sondierungsreisen deutscher Hochschulen und Forschungseinrichtungen ("Travelling Conferences") zum Aufbau von Kooperationen mit Partnern im Südkaukasus, Zentralasien und der Mongolei, Bundesanzeiger vom 05.09.2017) in 2017.

\section{PhD Thesis}
\textbf{Internet Of Things.} The Internet of the present day is enriched with a huge amount of various devices and micro services. These devices and services operate for different purposes, yet work together to achieve common goals. The various types of sensors, actuators and devices are being developed to provide a range of services. These sensors and devices have created a new technological trend today. This trend has been named the "Internet of Things". 

In the framework of IoT, more and more devices are equipped with network connectivity to autonomously provide "smarter" services, forming the Internet of Things (IoT). Applications are wide-ranging, and have variously been termed "Smart X", including Smart Homes, Smart Factories (Industry 4.0), Smart Government, Smart City, Smart Grid, Smart Traffic Control, and many more.

Smart technologies employ sensors and actors to increase the efficiency of services and processes, including environmental sustainability, energy efficiency, mobility, health care, safety, and security. ICT helps to optimize the processes, while IoT provides the platform for managing a multitude of small sensor, actor devices, communication protocols and home servers.

The concept of smart technologies arises from the need to manage, automate, optimize and explore all aspects of daily life that could be improved and optimized by information technologies. The software paradigm IoT, being a core concept behind smart smart technologies, is largely perceived as a collection of interconnected "things" within smart technologies.

\textbf{Model-Driven Engineering.} The IoT-based smart applications are realized by interconnected systems of heterogeneous hardware, software, and embedded systems: these cyber-physical systems introduce new levels of complexity, requiring appropriate engineering methodologies to support formally rigorous software and systems development. \textbf{Model-Driven Engineering (MDE)} provides fitting foundations and is considered as an enabling technology for advancing smart technology applications.

MDE is the modern day approach of software system development which supports well-suited abstraction concepts for development activities. It intends to improve the productivity of the design and development, maintenance activities, and communication among various actors and stakeholders of a system. As the main concept in MDE, models are well-suited for designing, developing and producing large-scale software projects. MDE brings several main benefits such as a productivity boost, models become a single point of truth. Models are the main artifacts in MDE. They are well-suited for designing, developing and producing large software systems. Software models are the documentation and implementation of software systems.

\textbf{MDE Applied: Model-Driven IoT.} There are several domain-specific MDE approaches and tools for developing IoT-based architectures and applications. However, research in model-driven IoT is in its early stages and extended research in the field is required. MDE is not fully utilized for developing model-driven IoT approaches. This proposal focuses on the field of model-driven IoT for smart systems. Its objectives are manifold: (1) studying the state of the art in model-driven IoT, (2) extended research in model-driven IoT focusing on smart technologies, (3) applying MDE concepts to develop software architectures and platforms for IoT-based systems.

The core objectives of this proposal are manifold:
\begin{itemize}
\item[--] \textbf{Studying The State of Art.} This research initially intends to study the state of the art in model-driven IoT approaches in order to identify the MDE tools and approaches dedicated especially to develop the IoT-based smart systems. As long as these tools are open-source and their underlying concepts can be further developed and extended. 
\item[--] \textbf{Research.} The most important, challenging and long-running part of this research focuses on extended research in model-driven IoT focusing on the smart technologies. This research aims to utilize the full potential of MDE in developing advanced model-driven IoT systems, applications and architectures.
\item[--] \textbf{Application and validation.} Application area of Model-driven IoT is wide. For example, we can utilize Model driven languages to describe, analyze, maintain and evolve IoT systems. It makes easy to store and manage models of the IoT systems in a centralized repository. IoT models can save time and supply complexity by providing automatic code generation. Automatic code generation also reduces the risk of errors, which leads to the quality improvement. Also, it gives the opportunity to concentrate on adding more value. This research further focuses on applying a proposed approach to real-world applications as the proof of concept.
\item[--] \textbf{Cooperation with companies.} During the period of my research stay, I intend to get acquainted with the activity of the companies working in Germany, particularly in Oldenburg, in the field of IoT. As known [http://www.oldenburg.de/de/startseite/buergerservice/aktuelles/smart-city.html], Oldenburg city administration is working towards becoming a smart city.
\end{itemize}

\section{Motivation and Goals}
Today’s technological advancements leading to an increasingly interconnected world. We are using smarter devices and services in our everyday life. Internet of things (IoT) has become the new technological trend today. Therefore, I have chosen model-driven IoT as my PhD research area. I am currently working in the PhD topic entitled "Model-driven Internet of Things".

German universities, and particularly, Software Engineering Group at University of Oldenburg is one of the leading institutions in the field of software engineering and its model-driven aspects. Therefore, I want to do my PhD research in the Software Engineering Group at the University of Oldenburg. I am interested in the scientific works of this department, as long as they are in my interests. Additionally, both the University of Oldenburg and TUIT have signed a "Memorandum of Understanding" to setup and strengthen partnership and cooperation in Software 
Engineering and it’s application areas. 

I plan to expand my knowledge in the field of software engineering while visiting University of Oldenburg. In particular, to strengthen the theoretical knowledge of the foundations of modeling, model-driven software architectures, version-control systems, service-oriented architectures, micro-service architectures and modeling. These areas are the main foundations for the research field of model-driven IoT. The main purpose of research stay is to complete my research proposal in the field of Model-Driven IoT. Moreover, I plan to publish the scientific results of my research in different conferences and workshops, as well as discuss my research topic with various experts and scientists in the relevant field. And yet, if possible, I would like to establish contacts with German companies working in this field.

At the same time, I want to participate in current projects of Software Engineering group as a visitor. Participation in these projects will help me to enrich my practical skills in Software Engineering.

I am very keen to develop and expand existing cooperations between the Tashkent University of Information Technologies and the Carl von Ossietzky University of Oldenburg. The potential research fields consist of smart modeling, smart cities based on IoT and model-driven IoT.

\section{Work and Time Plan}
I intend to apply for four months research stay in Oldenburg. During this research stay, I will to pursue concrete work and time plan as follows:
\begin{itemize}
\item[--] \textbf{The First Month:} The First Month: In the beginning of my research stay, I would like to familiarize myself with the existing infrastructure and working environment in the Software Engineering Group. In this phase, I will start extended literature review in the field of Internet of Things and Model-driven Engineering. Moreover, I am very keen on learning German language. I want to start learning German in Language center of University of Oldenburg from the first month of the stay.
\item[--] \textbf{The Second Month:} After studying the state of the art in IoT and MDE, I will continue literature study in Model-Driven IoT to review and identify the research challenges. Meanwhile, I will learn how to write scientific papers.
\item[--] \textbf{The Third Month:} According to the research findings till this time, I will look for workshops and conferences to publish my research results.
\item[--] \textbf{The Fourth Month:} In the final phase of my research stay, I will finalize documentation of my research stay, eventually complete my research proposal which will be the part of my application for the DAAD full PhD scholarship. This document will be extended research proposal for continuing my PhD thesis under supervision of Prof. Winter.
\end{itemize}

Along the activities above, I will seek for developing contacts with other researchers and PhD students to discuss my topic.

\section{Necessity of Funding}
To develop my scientific work and establish strong ties with the department, I would like to visit the University of Oldenburg and closer get acquainted with my supervisor and Software Engineering group. This research stay will not be paid by TUIT. I need a funding to finance my research stay in Software Engineering group. The joint contact scholarship for foreign doctoral students by DAAD will be the best aid for me to achieve plans and results in this proposal. Because I have no other source of financial support except my own budget, it would be very helpful for my research stay in Oldenburg and develop closer cooperations with the research colleagues and to discuss further scientific activities with Prof. Andreas Winter in Oldenburg.
\end{document}